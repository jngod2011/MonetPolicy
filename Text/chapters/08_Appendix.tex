% % % % % % % % % % % % % % % % % % % % % % % % % % % % % %
% % % % % % 			Appendix			% % % % % % % %
% % % % % % % % % % % % % % % % % % % % % % % % % % % % % %	

\subsection{Exemplary newspaper articles}
%

% Example Article #1
\begin{figure}[!h]
	\centering
	\includegraphics[width=.5\textwidth]{chapters/tables_graphs/Example_article.png} 
	\caption{Example of an article that has been used for the sentiment determination.}
	\label{fig:ExArticle_1}
\end{figure}
%
% Example Article #2
\begin{figure}[!h]
	\centering
	\includegraphics[width=1\textwidth]{chapters/tables_graphs/Example_article_02.png} 
	\caption{Example of an article that has been used for the sentiment determination.}
	\label{fig:ExArticle_2}
\end{figure}
%

\newpage
\subsection{Part-of-Speech tags.}
%
\begin{table}[!h]
	\tiny
	\caption[The Penn English Treebank POS tagset]{The Penn English Treebank POS tagset \parencite{Marcus.1993}.}
	\label{tab:POS_tags}
	\centering
	\begin{tabular}{ll}
		\toprule
		CC& Coordinating conjunction \\
		CD& Cardinal number\\
		DT& Determiner\\
		EX& Existential \textit{there}\\
		FW& Foreign word\\
		IN& Preposition/subordinating conjunction\\
		JJ& Adjective\\
		JJR& Adjective, comparative\\
		JJS& Adjective, superlative\\
		LS& List item marker\\
		MD& Modal\\
		NN& Noun, singular or mass\\
		NNS& Noun, plural\\
		NNP& Proper noun, singular\\
		NNPS& Proper noun, plural\\
		PDT& Predeterminer\\
		POS& Possessive ending\\
		PRP& Personal pronoun\\
		PRP\$& Possessive pronoun\\
		RB& Adverb\\
		RBR& Adverb, comparative\\
		RBS& Adverb, superlative\\
		RP& Particle\\
		SYM& Symbol (mathematical or scientific)\\
		TO& \textit{to}\\
		UH& Interjection\\
		VB& Verb, base form\\
		VBD& Verb, past tense\\
		VBG& Verb, gerund/present participle\\
		VBN& Verb, past participle\\
		VBP& Verb, non-3rd person singular present \\
		VBZ& Verb, 3rd person singular present\\
		WDT& \textit{wh}-determiner\\
		WP& \textit{wh}-pronoun\\
		WP\$& Possessive \textit{wh}-pronoun\\
		WRB& \textit{wh}-adverb\\
		%& \\
		\bottomrule
	\end{tabular}
\end{table}
%
%
% Example POS tagged article
\begin{figure}[!h]
	\centering
	\includegraphics[width=1\textwidth]{chapters/tables_graphs/Example_article_POS.png} 
	\caption{Example of an article snippet with POS tags.}
	\label{fig:POStaggedArticle}
\end{figure}
%

\newpage
\subsection{Sentiment indicators for count-based evaluation}
%
\begin{table}[!h]
	\tiny
	\caption{Terms that have been defined as indicators for endogenous or exogenous sentiment.}
	\label{tab:sentiment_words}
	\centering
	\begin{tabular}{ll}
		\toprule
		Endogenous & Exogenous \\
		\midrule
		unchanged &	switch\\
		remain	&	unexpected\\
		steady	&	larger\\
		continue	&	sharp\\
		sustain	&	less\\
		prevent	&	policy change\\
		as a result	&	new\\
		economic recovery	&	surprise\\
		recovery	&	longer period\\
		slowdown	&	for a while\\
		downturn	&	despite\\
		recession	&	gradual\\
		crisis	&	signal\\
		slow growth	&	more/less aggressive\\
		high growth	&	tightening\\
		unemployment	&	loosening\\
		labor	&	cautious\\
		productivity	&	CPI\\
		real estate	&	pressure\\
		housing	&	productivity\\
		financial market	&	ahead\\
		natural catastrophe	&	pause\\
		hurricane	&	\\
		terrorist attack	&	\\	
		\bottomrule
	\end{tabular}
\end{table}
%

\newpage
\subsection{Additional regression results}
%

% Non-policy vs policy days QE
\input{chapters/tables_graphs/NPvsPdays_QE.tex}
%

% Endog vs Exog table QE
\input{chapters/tables_graphs/EndogvsExogdays_QE.tex}
%

% Non-policy vs policy days no crisis
\input{chapters/tables_graphs/NPvsPdays_short.tex}
%

% Endog vs Exog table no crisis
\input{chapters/tables_graphs/EndogvsExogdays_short.tex}
%
