% % % % % % % % % % % % % % % % % % % % % % % % % % % % % %
% % % % % % 			Methodology			% % % % % % % %
% % % % % % % % % % % % % % % % % % % % % % % % % % % % % %	

\subsection{Monetary Policy}

\dots

% FED target history, analyses about yield curve movements

\subsection{Computational Linguistics}

\dots

% why is this becoming more important and why can it be used for our purposes

\subsection{Opinion Mining}

\textcite{Turney.2002} introduces an unsupervised three-step learning algorithm to mark a review as positive or negative\footnote{Neutral sentiment is ignored in many cases as a clear distinction from the non-neutral sentiments is not possible.} where first, \dots

\textcite{Pang.2002}, on the other hand, applies and compares three supervised learning techniques to classify reviews, namely na\"{i}ve Bayes, maximum entropy and support vector machines. 

% identification of relevant articlesd not done, either first page... or just one column.... arbitrary! might lose info

% unigrams and bigrams