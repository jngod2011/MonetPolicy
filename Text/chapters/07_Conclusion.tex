% % % % % % % % % % % % % % % % % % % % % % % % % % % % % %
% % % % % % 			Conclusion 			% % % % % % % %
% % % % % % % % % % % % % % % % % % % % % % % % % % % % % %	

\textcite{Ellingsen.2001} introduce an economic model which relates monetary policy actions to yield curve movements and succeeds in implementing the empirical feature of yield curve parallel shifts after a target rate adjustment has been performed which was interpreted by market participants as a reaction to new information about the economy, referred to as endogenous. If the change is instead interpreted as exogenous shift in preference by the Fed, the model mimics observations that show a tilt in the yield curve. While \textcite{Ellingsen.2003} find empirical evidence for this theoretical model by first classifying target rate adjustments by analysing newspaper articles by hand, this thesis automates their procedure, using state of the art text mining and machine learning procedures. 

Even though a variety of intuitive and simulation checks have been performed on the alternative classification task, I fail to replicate their findings which is mainly due to the limited amount of variation in the target rate adjustments as well as unprecedented interventions by the monetary authorities during the sample period. Conversely, this implies that the model does not account for alternative policy actions that go beyond adjusting the target rate and falls short of covering the complete range of possible monetary policy actions and their impact on market interest rates.

Furthermore, the techniques applied are standard methods which are well discussed in the computer science literature and applied on a variety of topics. This implies that their utility has been proven in many fields but conversely, their level of specialisation is quite low. Since causality cannot be established as transparently as with common econometric means, they have to be applied with great care in such a specialised setting and a lot of manual labour has to be put into the selection of informative training sets. An additional issue with respect to the feasibility of the technique might be that the classification algorithm could not be applied to the sample used by \textcite{Ellingsen.2003} for which only few articles were available which would have supported its accuracy.

One extension that might mitigate this issue is applying bootstrapping to the training sample in order to get more meaningful data to train the algorithms. Since this would have been beyond the scope of this thesis, it is left for further discussions alongside the application of more sophisticated supervised learning algorithms as well as clustering and unsupervised methods. On top of that, the elastic database \parencite{Elastic.2015} offers a very convenient way to extend a dataset consisting of articles through its implemented feature to scan existing data with respect to similarities. By utilising this feature, re-running the analysis of this thesis on a sample period with normal interest rates will be a great asset in evaluating the accuracy of the economic model at hand as well as the applied methodology.
